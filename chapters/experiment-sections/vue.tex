% + Pinia for global state management (officially supported)
% + code is confined in one file (custom file type) and easily readable
% + boilerplate free (besides setup which is automated by Vite)
% + big ecosystem
% + great documentation with interactive walkthrough
% + gets used a lot (including in enterprises)
% + performant out of the box
% + professional support is available through 3rd parties
% + easy state management, even in complex components
% + backed by (financially stable) open source contributors
% ~ uses VDOM, but makes compile time optimizations for this and for its state management
% ~ client rendered (Nuxt or VuePress solve this)
% - nested generics can cause troubles

\section{Vue}
\label{sec:vue}

\subsection{Overview}
\label{subsec:vue:overview}

Vue, created by Evan You and maintained by the open source community. It has been financially sustainable since 2016 funded through sponsorships. It has been built to be approachable by using standard \acrshort{html}, \acrshort{css} and \acrshort{js}, greatly documented, performant updates by using compile time optimizations \cite{vue:home}. Just like Svelte \textit{(\ref{sec:svelte})} it uses its own file type which is very comparable in structure, the only difference is that unlike with Svelte you need to wrap your \acrshort{html} contents with a template tag \cite{svelte:components, vue:template_syntax}. This comes with the benefit of being almost boilerplate free. Also like React \textit{(\ref{sec:react})} it uses the \acrshort{vdom} to manage its reactivity logic \cite{react:vdom,vue:rendering_mechanism}. The Vue team also maintains two additional packages that are useful for big applications, these are the Vue Router and Pinia, which is a global state management system comparable to Redux \textit{(\ref{subsec:redux})}, but keeps the same state management logic as normal Vue state management processing \cite{vue:router,vue:pinia}. Making it extremely fast to learn and easy to use.

\itodo{vue: continue writing this}

\subsection{Strengths}
\label{subsec:vue:strengths}
\begin{itemize}
    \item strength! \todo{vue: write strengths}
\end{itemize}

\subsection{Weaknesses}
\label{subsec:vue:weaknesses}
\begin{itemize}
    \item weakness! :( \todo{vue: write weaknesses}
\end{itemize}