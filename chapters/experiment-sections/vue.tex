\section{Vue}
\label{sec:vue}

\subsection{Overview}
\label{subsec:vue:overview}

Vue, created by Evan You and maintained by the open-source community, has been financially sustainable since 2016 through sponsorships \cite{vue:sponsor,vue:faq}. It is built to be approachable, utilizing standard \acrshort{html}, \acrshort{css}, and \acrshort{js}, and is extensively documented. Its updates are performant due to compile-time optimizations \cite{vue:home}.

Like Svelte \textit{(\ref{sec:svelte})}, Vue uses its own file type, which is similar in structure. The primary difference is that in Vue, unlike Svelte, you need to wrap your \acrshort{html} content with a template tag \cite{svelte:components, vue:template_syntax}. This approach almost eliminates boilerplate code.

Similar to React \textit{(\ref{sec:react})}, Vue uses the \acrshort{vdom} \textit{(\ref{subsec:vdom})} to manage its reactivity logic \cite{react:vdom,vue:rendering_mechanism}. Additionally, the Vue team maintains two packages beneficial for large applications: Vue Router and Pinia. Pinia serves as a global state management system, comparable to Redux \textit{(\ref{subsec:redux})}, but retains the same state management logic as standard Vue processes, making it extremely fast to learn and easy to use \cite{vue:router,vue:pinia}.

Furthermore, Vue's state management system is easy to use and scalable, without the confusing side effects seen in React.

\subsubsection{Challenges and Considerations}
\label{subsec:vue:challenges}

The \acrshort{vdom} introduces a memory overhead in the browser, which is a concern not present in other solutions that do not utilize the \acrshort{vdom}. Despite several compile-time optimizations, this overhead remains an issue. Additionally, there are no officially supported \acrshort{ssr} solutions for \acrshort{vdom}. However, popular and stable alternatives like Nuxt.js and VuePress are available. Lastly, there have been issues with nested generics passing through slots not correctly detecting types, but this is a narrow problem that will likely be resolved with a future patch.

\subsection{Strengths}
\label{subsec:vue:strengths}
\begin{itemize}
    \item great official global state management library \textit{(Pinia)}
    \item great official router library \textit{(Vue Router)}
    \item custom file type that is easy to learn and readable
    \item great documentation with an interactive walkthrough
    \item lots of usage, including in enterprises
    \item big ecosystem
    \item performant out of the box due to compile time optimizations
    \item easy to learn and maintainable state management system
    \item financially stable and backed by Vercel
    \item boilerplate free
\end{itemize}

\subsection{Weaknesses}
\label{subsec:vue:weaknesses}
\begin{itemize}
    \item \acrshort{vdom} memory overhead
    \item no out of the box \acrshort{ssr} solution
    \item issues with nested generics to child params
\end{itemize}


\subsection{Scores}
\label{subsec:vue:scores}

\begin{table}[H]
    \centering
    \begin{tabular}{|l|l|}
        \hline
        \textbf{Method}            & \textbf{Score}                                    \\
        \hline
        Easiness/speed to learn    & 1                                                 \\ \hline
        State management           & 1                                                 \\ \hline
        Boilerplate                & 1 \textit{(components: 1, state management: 1)}   \\ \hline
        \acrshort{api} integration & 0 \textit{(default browser fetch \acrshort{api})} \\ \hline
    \end{tabular}
    \caption{complexity}
    \label{tab:vue:complexity}
\end{table}

\begin{table}[H]
    \centering
    \resizebox{\columnwidth}{!}{
        \begin{tabular}{|l|l|}
            \hline
            \textbf{Method}                          & \textbf{Score}                                                                                          \\
            \hline
            Community                                & 0.8 \textit{(\autoref{tab:metrics:community})}                                                          \\ \hline
            Professional Support                     & 1                                                                                                       \\ \hline
            Documentation (interactive walkthrough?) & 1                                                                                                       \\ \hline
            Ecosystem                                & 0.775 \textit{(quality: 0.85, size: 0.7; \autoref{tab:metrics:vue:ratings} \autoref{tab:metrics:size})} \\ \hline
            Usage by other enterprises               & 1                                                                                                       \\ \hline
            Complexity                               & 0.75 \textit{(\autoref{tab:vue:complexity})}                                                            \\ \hline
            Server Side Rendered (SSR)               & 0.5                                                                                                     \\ \hline
        \end{tabular}
    }
    \caption{Vue general scores}
    \label{tab:vue:scores}
\end{table}