% gebruikt DOM 
% veel boilerplate
% veel built in/gestandariseerde dependencies
% andere terminologie
% client side by default heeft SSR functies ingebouwd
% development mode laat je niet toe om van taal te veranderen/i18n te testen omdat hiervoor een apparte build nodig is
% groot ecosysteem
% complexere syntax
% goede docs
% standaard typescript en html bestanden, maar eigen templating language in de html
% je kan je scope voor de style zelf kiezen
% backged by google
% veel enterprise gebruik
% templating is minder krachtig dan bv vue, ook is het complexer om simpele dingen te doen (eg slot)
% CLI application to generate basic boilerplate
% je mag geen dingen vergeten voor legacy redenen (eg standalone op true zetten)
% onderdelen voelen te expliciet aan (eg @input required: type) ipv van het type af te leiden
% update process is stabiel (wordt getest op meer dan 100k componenten @ google)
% ingebouwde routing framework is stabiel
% ipv componenten te gebruiken bind je op css selectoren, wat je dus een veel krachtigere toegang geeft
% state management is complex (twee manieren van state beheren, adhv signals en zonder, maar je kan alleen maar singnals meegeven als het dat verwacht)

\section{Angular}
\label{sec:angular}

\subsection{Overview}
\label{subsec:angular:overview}
\itodo{write this}

\subsection{Strengths}
\label{subsec:angular:strengths}
\begin{itemize}
    \item strength! \todo{write strengths}
\end{itemize}

\subsection{Weaknesses}
\label{subsec:angular:weaknesses}
\begin{itemize}
    \item weakness! :( \todo{write weaknesses}
\end{itemize}