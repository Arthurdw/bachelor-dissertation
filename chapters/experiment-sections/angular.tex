% + no VDOM overhead
% + lots of built in/standard dependencies
% + client side, but has SSR baked in
% + big ecosystem
% + dependency injection
% + great documentation
% + standard typescript and html files (custom templating language in the html)
% + scope of css is easily determinable
% + backed by Google
% + lots of usage (especially in enterprises)
% + CLI application to generate the basic boilerplate
% + stable updating process
% + build in routing framework is stable and powerful
% + instead of components css selectors are used, this gives you much more power to create advanced component behavior
% - state management is complex (eg, to ways of managing state, with and without signals; but signals can only expect signals as argument)
% - templating language is less powerful compared to the other solutions (eg argument passing to slots)
% - complex syntax
% - i18n can not change in development mode (a separate build is required for each language)
% - lots of boilerplate (mostly for legacy purposes)
% - doesn't always use the type systems (eg for an @input you have to explicitly set required: true, instead of deriving this form the typescript type)

\section{Angular}
\label{sec:angular}

\subsection{Overview}
\label{subsec:angular:overview}

Angular is a \gls{framework} created by Google. By default it has many features out of the box, like routing, state management, dependency injection, custom \acrshort{http} client, \dots\ - which makes a standardized approach for developing applications. To manage data Angular uses two way bindings and the \acrshort{dom}.

\itodo{write over updating prrocess}

\subsection{Strengths}
\label{subsec:angular:strengths}
\begin{itemize}
    \item mediocre documentation
    \item standard typescript and \acrshort{html} files (custom templating language in \acrshort{html})
    \item \acrshort{css} scope can be chosen with ease
    \item big Community \textit{(over 40,000 members on Discord)}
    \item \acrshort{cli} application to help generate boilerplate
    \item standardized approach/libraries
    \item uses \acrshort{css} selectors instead of components, providing more flexibility to the developer.
\end{itemize}

\subsection{Weaknesses}
\label{subsec:angular:weaknesses}
\begin{itemize}
    \item Uses different terminology than all other frameworks which makes switching more difficult.
\end{itemize}