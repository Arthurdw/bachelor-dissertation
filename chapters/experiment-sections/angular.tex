\section{Angular}
\label{sec:angular}

\subsection{Overview}
\label{subsec:angular:overview}

Angular, created by Google, is a flexible web framework valued for its strength and wide range of tools. This section explores its key features and considerations. Angular includes many pre-built dependencies, making development smoother and saving time on initial setup. A notable feature is its support for \acrfull{ssr} \cite{angulardev:ssr}, which increases deployment options. The framework is thoroughly documented, suitable for developers at any experience level. Angular utilizes standard TypeScript and \acrshort{html} files, effortlessly blending a custom templating language into \acrshort{html} \cite{angulardev:template_syntax}.

\subsubsection{Styling and Project Setup}

Angular's CSS scoping mechanism ensures clarity in styling scope, thus enhancing code maintainability \cite{angulardev:styling_components}. Additionally, its CLI application accelerates project setup by efficiently generating boilerplate code \cite{angulardev:cli}.

\subsubsection{Challenges and Considerations}
\label{subsec:angular:challenges}

However, Angular faces challenges, particularly in state management. It provides two methods for managing state, through two-way bindings and signals, which can be confusing and are not interoperable \cite{angulardev:managing_dynamic_data, angulardev:signals}.

Angular offers a powerful templating language, but it may fall short in certain areas compared to other options. For instance, passing arguments to slots/content projection can be less intuitive for new developers, potentially increasing the learning curve \cite{angulardev:ng_container}. Moreover, Angular's approach to internationalization requires separate builds for each language, limiting flexibility during development since it lacks built-in support for language switching \cite{angulardev:i18n}.

Angular entails a significant amount of boilerplate, primarily due to legacy considerations \cite{angulardev:directive_composition_api}. Moreover, it lacks consistency in utilizing type systems, necessitating explicit declarations for certain features like input validation \cite{angulardev:inputs, typescript:objects}.

\subsection{Update Process}

Angular boasts a reliable updating process, supported by detailed migration guides for smooth transitions across major versions. Major updates occur every six months, typically accompanied by 1 to 3 minor releases. Furthermore, patch releases are rolled out nearly every week, enhancing the platform's stability and functionality \cite{angulardev:versioning_and_releases}.

\subsection{Unique Features}

Angular employs \acrshort{css} selectors instead of tags, granting developers more control over component behavior and implementations \cite{angulardev:component_selectors}. Additionally, it implements dependency injection, a beneficial design pattern \cite{stackify:dependency_injection, angulardev:dependency_injection}.

\subsection{Strengths}
\label{subsec:angular:strengths}
\begin{itemize}
    \item no \acrshort{vdom} overhead
    \item great documentation
    \item standard typescript and \acrshort{html} files (custom templating language in \acrshort{html})
    \item \acrshort{css} scope can be chosen with ease
    \item big community \textit{(over 40,000 members on Discord)}
    \item \acrshort{cli} application to help generate boilerplate
    \item stable standardized approach/libraries for many features
    \item uses \acrshort{css} selectors instead of tags for components, providing more flexibility to the developer component usage
    \item baked in \acrshort{ssr} support
    \item you can easily set the scope for styles
    \item stable and consistent release process \cite{angulardev:versioning_and_releases}
    \item lots of enterprise usage
    \item consistent and stable updating process
\end{itemize}

\subsection{Weaknesses}
\label{subsec:angular:weaknesses}
\begin{itemize}
    \item uses different terminology than all other frameworks which makes switching more difficult
    \item \acrshort{hmr} doesn't work great/smoothy out of the box
    \item you are unable to change the language for \acrshort{i18n} in development mode as it requires a separate build for each language
    \item state management is complex \textit{(\autoref{subsec:angular:challenges})}
    \item requires a lot of boilerplate
    \item templating language is less powerful compared to other solutions \textit{(\autoref{subsec:angular:challenges})}
    \item requires explicit declarations that could be derived from TypeScript
\end{itemize}