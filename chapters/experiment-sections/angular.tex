\section{Angular}
\label{sec:angular}

\subsection{Overview}
\label{subsec:angular:overview}

Angular, a web framework developed by Google, is esteemed for its flexibility and extensive toolset. This section delves into its primary features and considerations. With numerous pre-built dependencies, Angular streamlines development and reduces setup time. Notably, it supports \acrfull{ssr} \cite{angulardev:ssr}, expanding deployment possibilities. The framework boasts comprehensive documentation, catering to developers of all skill levels. Leveraging standard TypeScript and \acrshort{html} files, Angular seamlessly integrates a custom templating language into \acrshort{html} \cite{angulardev:template_syntax}.

\subsubsection{Styling and Project Setup}

Angular's CSS scoping mechanism ensures clarity in styling scope, thus enhancing code maintainability \cite{angulardev:styling_components}. Additionally, its \acrshort{cli} application accelerates project setup by efficiently generating boilerplate code \cite{angulardev:cli}.

\subsubsection{Challenges and Considerations}
\label{subsec:angular:challenges}

Angular faces challenges, particularly in state management. It provides two methods for managing state: through two-way bindings and signals, which can be confusing and are not interoperable \cite{angulardev:managing_dynamic_data, angulardev:signals}.

While Angular offers a powerful templating language, it may fall short in certain areas compared to other solutions. For instance, passing arguments to slots/content projection can be less intuitive for new developers, potentially increasing the learning curve \cite{angulardev:ng_container}. Moreover, Angular's approach to internationalization requires separate builds for each language, limiting flexibility during development since it lacks built-in support for language switching \cite{angulardev:i18n}.

Angular entails a significant amount of boilerplate, primarily due to legacy considerations \cite{angulardev:directive_composition_api}. Moreover, it lacks consistency in utilizing type systems, necessitating explicit declarations for certain features like input validation \cite{angulardev:inputs, typescript:objects}.

\subsection{Update Process}

Angular boasts a reliable updating process, supported by detailed migration guides for smooth transitions across major versions. Major updates occur every six months, typically accompanied by 1 to 3 minor releases. Furthermore, patch releases are rolled out nearly every week, enhancing the platform's stability and functionality \cite{angulardev:versioning_and_releases}.

\subsection{Unique Features}

Angular employs \acrshort{css} selectors instead of tags, granting developers more control over component behavior and implementations \cite{angulardev:component_selectors}. Additionally, it implements dependency injection, a beneficial design pattern \cite{stackify:dependency_injection, angulardev:dependency_injection}.

\subsection{Strengths}
\label{subsec:angular:strengths}
\begin{itemize}
    \item no \acrshort{vdom} overhead
    \item great documentation
    \item standard typescript and \acrshort{html} files (custom templating language in \acrshort{html})
    \item \acrshort{css} scope can be chosen with ease
    \item big community \textit{(over 40,000 members on Discord)}
    \item \acrshort{cli} application to help generate boilerplate
    \item stable standardized approach/libraries for many features
    \item uses \acrshort{css} selectors instead of tags for components, providing more flexibility to the developer component usage
    \item baked in \acrshort{ssr} support
    \item you can easily set the scope for styles
    \item stable and consistent release process \cite{angulardev:versioning_and_releases}
    \item lots of enterprise usage
    \item consistent and stable updating process
\end{itemize}

\subsection{Weaknesses}
\label{subsec:angular:weaknesses}
\begin{itemize}
    \item uses different terminology than all other frameworks which makes switching more difficult
    \item \acrshort{hmr} doesn't work great/smoothy out of the box
    \item you are unable to change the language for \acrshort{i18n} in development mode as it requires a separate build for each language
    \item state management is complex \textit{(\autoref{subsec:angular:challenges})}
    \item requires a lot of boilerplate
    \item templating language is less powerful compared to other solutions \textit{(\autoref{subsec:angular:challenges})}
    \item requires explicit declarations that could be derived from TypeScript
\end{itemize}


\subsection{Scores}
\label{subsec:angular:scores}

\begin{table}[H]
    \centering
    \begin{tabular}{|l|l|}
        \hline
        \textbf{Method}            & \textbf{Score}                                  \\
        \hline
        Easiness/speed to learn    & 0.5                                             \\ \hline
        State management           & 0                                               \\ \hline
        Boilerplate                & 0 \textit{(components: 0, state management: 0)} \\ \hline
        \acrshort{api} integration & 0.5 \textit{(has custom HTTP handler)}          \\ \hline
    \end{tabular}
    \caption{complexity}
    \label{tab:angular:complexity}
\end{table}

\itodo{anglar: relative performance}
\itodo{anglar: calculate community}
\itodo{anglar: calculate ecosystem}
\itodo{anglar: calculate added size}

\begin{table}[H]
    \centering
    \resizebox{\columnwidth}{!}{
        \begin{tabular}{|l|l|}
            \hline
            \textbf{Method}                          & \textbf{Score}                                                                                   \\
            \hline
            Community                                & 0.6 \textit{(\autoref{tab:metrics:community})}                                                   \\ \hline
            Professional Support                     & 1                                                                                                \\ \hline
            Documentation (interactive walkthrough?) & 1                                                                                                \\ \hline
            Ecosystem                                & - \textit{(quality: -, size: -; \autoref{tab:metrics:react:ratings} \autoref{tab:metrics:size})} \\ \hline
            Usage by other enterprises               & 1                                                                                                \\ \hline
            \textit{(added)} Size of solution        & -                                                                                                \\ \hline
            Relative performance of solution         & -                                                                                                \\ \hline
            Complexity                               & 0.25 \textit{(\autoref{tab:angular:complexity})}                                                 \\ \hline
            Server Side Rendered (SSR)               & 1                                                                                                \\ \hline
        \end{tabular}
    }
    \caption{React general scores}
    \label{tab:react:scores}
\end{table}