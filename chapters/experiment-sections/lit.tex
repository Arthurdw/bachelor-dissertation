\section{Lit}
\label{sec:lit}

\subsection{Overview}
\label{subsec:lit:overview}

Lit, developed by Google, is a library for browser native web components, emphasizing simplicity and versatility. Unlike libraries employing \acrshort{vdom}, Lit is lightweight and adaptable, compatible with any web solution, including plain \acrshort{html}. It minimizes boilerplate and utilizes straightforward \acrshort{html} alongside \acrshort{js} or \acrshort{ts}. Key features include reactive state management, scoped styles, and a declarative templating system \cite{lit:intro}.

Its primary objective is to facilitate the creation of shareable components and design systems for seamless integration across different solutions. Additionally, Lit can enhance basic \acrshort{html} sites progressively.

\subsubsection{Challenges and considerations}

Web components, unlike other solutions, aren't widely adopted, leading to a smaller ecosystem for Lit. The documentation also needs improvement \cite{github:issue_lit}. Additionally, Lit currently lacks \acrshort{ssr} capability and relies on Shadow \acrshort{dom} \textit{(see \ref{subsec:lit:shadow_dom})}. Integration with \acrshort{ide}s is subpar, offering inaccurate linting and suggestions \cite{lit:ssr, designsystemcentral:web_components, mitosis:overview}.

\subsection{Shadow DOM}
\label{subsec:lit:shadow_dom}

The Shadow \acrshort{dom} is a web standard that creates encapsulated components in web applications. It hides \acrshort{html}, \acrshort{css}, and \acrlong{js} within a scoped boundary. By attaching a separate \acrshort{dom} tree to an element, it shields internal structure and styles, preventing conflicts and unintended manipulation from external sources. This isolation fosters reusable, modular components, enhancing maintainability and code reusability in web development. \cite{mdn:shadow_dom} However, challenges exist with assistive technologies, developer experience, and \acrshort{seo} integration. \cite{designsystemcentral:web_components, manuel:shadow_dom}

\subsection{Strengths}
\label{subsec:lit:strengths}
\begin{itemize}
    \item no \acrshort{vdom} overhead
    \item uses browser native \acrshort{api}'s \textit{(web components)}
    \item not much boilerplate
    \item uses plain \acrshort{html}, \acrshort{js}/\acrshort{ts}
    \item compatible with any web solution
    \item backed by Google
    \item Shadow \acrshort{dom} ensures that components behave consistent
\end{itemize}

\subsection{Weaknesses}
\label{subsec:lit:weaknesses}
\begin{itemize}
    \item \acrshort{ssr} is experimental
    \item mediocre ecosystem
    \item mediocre documentation
    \item Shadow \acrshort{dom} drawbacks can interfere with development simplicity
    \item \acrshort{ide} integration lacks compared to other solutions
    \item calls are by reference which changes the \acrshort{js} \texttt{this} context
\end{itemize}


\subsection{Scores}
\label{subsec:lit:scores}

\begin{table}[H]
    \centering
    \begin{tabular}{|l|l|}
        \hline
        \textbf{Method}            & \textbf{Score}                                       \\
        \hline
        Easiness/speed to learn    & 0.5                                                  \\ \hline
        State management           & 0.5                                                  \\ \hline
        Boilerplate                & 0.75 \textit{(components: 0.5, state management: 1)} \\ \hline
        \acrshort{api} integration & 0 \textit{(default browser fetch \acrshort{api})}    \\ \hline
    \end{tabular}
    \caption{complexity}
    \label{tab:lit:complexity}
\end{table}

\begin{table}[H]
    \centering
    \resizebox{\columnwidth}{!}{
        \begin{tabular}{|l|l|}
            \hline
            \textbf{Method}                          & \textbf{Score}                                                                                          \\
            \hline
            Community                                & 0.3 \textit{(\autoref{tab:metrics:community})}                                                          \\ \hline
            Professional Support                     & 0                                                                                                       \\ \hline
            Documentation (interactive walkthrough?) & 1                                                                                                       \\ \hline
            Ecosystem                                & 0.465 \textit{(quality: 0.63, size: 0.3; \autoref{tab:metrics:lit:ratings} \autoref{tab:metrics:size})} \\ \hline
            Usage by other enterprises               & 0.5                                                                                                     \\ \hline
            Complexity                               & 0.44 \textit{(\autoref{tab:lit:complexity})}                                                            \\ \hline
            Server Side Rendered (SSR)               & 0.5                                                                                                     \\ \hline
        \end{tabular}
    }
    \caption{lit general scores}
    \label{tab:lit:scores}
\end{table}

Total score: 3.205
\hfill
Percentile score: 0.458