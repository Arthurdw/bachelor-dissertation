% + no VDOM
% + browser native api's (web components)
% + not much boilerplate
% + plain html, js/ts
% + create components for-/compatible with- any web framework
% + Backed by Google
% ~ client side (SSR is experimental)
% ~ mediocre ecosystem
% ~ mediocre documentation (as result: <https://github.com/lit/lit/issues/4617>, and <https://github.com/lit/lit/issues/4618>)
% ~ uses the shadow DOM
% - IDE integration is suboptimal compared to other frameworks
% - calls are a reference by default, which binds another `this` to the call. This can cause confusing behavior.

\section{Lit}
\label{sec:lit}

\subsection{Overview}
\label{subsec:lit:overview}

Lit is a library developed by Google on the browser native web components. This makes it a lightweight solution without \acrshort{vdom} that can be used with any web solution, even with plain \acrshort{html}. It also does not require much boilerplate and uses plain \acrshort{html} in combination with \acrshort{js} or \acrshort{ts}. Out of the box it provides reactive state, scoped styles and a declarative templating system. \cite{lit:intro}

It's main goal is to develop sharable components or design systems for its cross solution usage ability. It can also be used to progressively enhance basic \acrshort{html} sites.

\subsubsection{Challenges and considerations}

Due to web components not being as popular as other solutions, lit does not enjoy a big ecosystem. The documentation is also mediocre and has room for improvement \cite{github:issue_lit}. On top of that does it not currently have the ability to do \acrshort{ssr}, and makes use of the Shadow Dom \textit{(see \ref{subsec:lit:shadow_dom})}. \acrshort{ide} integration is also suboptimal providing inaccurate linting and suggestions. \cite{lit:ssr, designsystemcentral:web_components, mitosis:overview}

\subsection{Shadow DOM}
\label{subsec:lit:shadow_dom}

\itodo{READ article https://www.matuzo.at/blog/2023/pros-and-cons-of-shadow-dom}

bla bla \cite{designsystemcentral:web_components}

\subsection{Strengths}
\label{subsec:lit:strengths}
\begin{itemize}
    \item strength! \todo{write strengths}
\end{itemize}

\subsection{Weaknesses}
\label{subsec:lit:weaknesses}
\begin{itemize}
    \item weakness! :( \todo{write weaknesses}
\end{itemize}


\subsection{Scores}
\label{subsec:lit:scores}

\itodo{lit: fill in all}

\begin{table}[H]
    \centering
    \begin{tabular}{|l|l|}
        \hline
        \textbf{Method}            & \textbf{Score}                                  \\
        \hline
        Easiness/speed to learn    & 0.5                                             \\ \hline
        State management           & 0                                               \\ \hline
        Boilerplate                & 0 \textit{(components: 0, state management: 0)} \\ \hline
        \acrshort{api} integration & 0.5 \textit{(has custom HTTP handler)}          \\ \hline
    \end{tabular}
    \caption{complexity}
    \label{tab:lit:complexity}
\end{table}

\itodo{lit: do all}
\itodo{lit: relative performance}
\itodo{lit: calculate added size}

\begin{table}[H]
    \centering
    \resizebox{\columnwidth}{!}{
        \begin{tabular}{|l|l|}
            \hline
            \textbf{Method}                          & \textbf{Score}                                                                                         \\
            \hline
            Community                                & 0.6 \textit{(\autoref{tab:metrics:community})}                                                         \\ \hline
            Professional Support                     & 1                                                                                                      \\ \hline
            Documentation (interactive walkthrough?) & 1                                                                                                      \\ \hline
            Ecosystem                                & 0.81 \textit{(quality: 0.82, size: 0.8; \autoref{tab:metrics:lit:ratings} \autoref{tab:metrics:size})} \\ \hline
            Usage by other enterprises               & 1                                                                                                      \\ \hline
            \textit{(added)} Size of solution        & -                                                                                                      \\ \hline
            Relative performance of solution         & -                                                                                                      \\ \hline
            Complexity                               & 0.25 \textit{(\autoref{tab:lit:complexity})}                                                           \\ \hline
            Server Side Rendered (SSR)               & 1                                                                                                      \\ \hline
        \end{tabular}
    }
    \caption{lit general scores}
    \label{tab:lit:scores}
\end{table}