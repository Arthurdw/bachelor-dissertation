\section{React}
\label{sec:react}

\subsection{Overview}
\label{subsec:react:overview}

React is a \gls{library} \cite{lagadritu:limitations_react} created by Meta \textit{(originally known as Facebook)}. Its recommended usage is in combination with \acrshort{jsx} \cite{react:introduction_jsx} \textit{(see \ref{subsec:jsx})}.
The library is primarily intended for the render layer and does not include native support for features such as routing and \acrshort{i18n}. However, this does not mean that you cannot easily incorporate these features, as the library is part of a vast community ecosystem that includes many high-quality packages that specialize in various areas.

React uses a different approach than plain \acrlong{js} by utilizing the \acrshort{vdom} \cite{react:vdom} \textit{(see \ref{subsec:vdom})} to manage content instead of directly manipulating the \acrshort{dom}. It attempts to detect changes using the reconciliation algorithm \textit{(see \ref{subsec:react:reconciliation_algorithm})} in the browser at runtime \cite{react:reconciliation}.

\subsection{JSX}
\label{subsec:jsx}

The rendering logic often gets tightly coupled with other \acrshort{ui} logic. Instead of separating things by putting markup and logic in separate files, we can achieve our separation of concerns \cite{enwiki:separation_of_concerns} by creating loosely coupled units called \textsl{components}. These components should ideally be pure, making the logic predictable, testable, and allowing us to make render optimizations like memoization \cite{enwiki:pure_function, react:keeping_components_pure}.

We can use \acrfull{jsx}, which is neither \acrshort{js} nor a string. Instead, it combines \textit{(as the name implies)} \acrshort{xml}/\acrshort{html} syntax with \acrshort{js} capabilities \textit{(demonstrated in \autoref{lst:jsx_interpolation})}. We can easily create components by defining a method that returns \acrshort{jsx}, essentially currying \cite{enwiki:currying} its context \textit{(arguments, state, and more; demonstrated in \autoref{lst:jsx_component})}.

\lstinputlisting[
    label={lst:jsx_interpolation},
    caption=Simple \acrshort{jsx} interpolation \cite{react:introduction_jsx, meta:jsx}
]{code/jsx_interpolation.tsx}

\lstinputlisting[
    label={lst:jsx_component},
    caption=Simple \acrshort{jsx} component
]{code/jsx_component.tsx}

\subsection{Update Process}
\label{subsec:react:update_process}

The update process of React is highly reliable. It involves testing the entire React-meta codebase, which comprises over 50,000 components, to determine if deprecating a method requires many changes. Only after this testing, does the React team decide if deprecation is necessary. If it is, they release a warning to the open-source community, which remains for one version. After that, the deprecated item is completely removed. In case many changes are needed to address the deprecation warning, scripts are built to make the migration as automatic as possible \cite{react:design_principles}.

\subsection{Virtual DOM}
\label{subsec:vdom}

The \acrfull{vdom} is a mirrored version of the real \acrshort{dom}. Represented as in-memory objects \textit{(eg. \autoref{lst:vdom_representation})} which can easily be traversed \textit{(as no \acrshort{dom} needs to be parsed)}, checked for changes, and used for other optimizations. For example, if a type of a \acrshort{vdom} element is changed it will tear down the old tree and rebuild the tree from scratch, but if the type is the same it will only update the attributes. Or if a key is set the reconciler can easily detect what items need to update  \textit{(\ref{subsec:react:reconciliation_algorithm}, \cite{react:reconciliation, acdlite:react_fiber_architecture})}.

Although the \acrshort{vdom} incurs more overhead as the browser has to keep the entirety in memory, it offers greater flexibility for the reconciler. For instance, the React reconciler can not only process the \acrshort{dom} but also native iOS and Android displays \textit{(with React Native)} \cite{acdlite:react_fiber_architecture}. Additionally, the \acrshort{vdom} enables more unique optimizations such as the pull technique instead of push, which allows the prioritization of user interactions over background tasks \cite{react:design_principles}. Moreover, it allows the renders to be batched instead of each one being its own operation.

\lstinputlisting[
    label={lst:vdom_representation},
    caption=\acrshort{json} representation of \acrshort{vdom} element \cite{react:components_elements_and_instances, react:react_basic}
]{code/vdom.json}

\lstinputlisting[
    language=xml,
    label={lst:vdom_representation_html},
    caption=\acrshort{html} equivalent of \autoref{lst:vdom_representation}
]{code/vdom.html}

\subsection{Reconciliation Algorithm}
\label{subsec:react:reconciliation_algorithm}

There are generic solutions to the algorithmic problem of diffing and transforming one tree into another. However, the existing algorithms are expensive at $O(n^3)$ \cite{bille2005survey}. Because this is too expensive for a web framework, the react reconciler implements a $O(n)$ algorithm based on two assumptions \cite{react:reconciliation}.

\begin{enumerate}
    \item \say{Two elements of different types will produce different tries.} which is why if the type is different the tear will be torn down.
    \item \say{The developer can hint at which child elements may be stable across different renders with a \textsl{key} prop.}
\end{enumerate}

\subsection{State Management}
\label{subsec:react:state_management}

State management in React relies on hooks, which are specialized functions. These hooks serve specific purposes within React components. Unlike traditional JavaScript assignments, manipulating state in React requires the use of hooks and their associated methods. Additionally, React enforces immutability, meaning once state is set, it cannot be directly changed. This immutability adds complexity to learning React, as it imposes restrictions on how actions can be performed.

When a state is updated in React, the reconciler detects these changes and initiates a refresh of the entire component in the next tick. To optimize performance and avoid unnecessary refreshes, it is recommended to:

\begin{itemize}
    \item Split components into smaller, more manageable pieces.
    \item Minimize side effects within components.
    \item Utilize memoization where applicable.
\end{itemize}

By following these practices, developers can ensure efficient state management and enhance the performance of React applications \cite{enwiki:pure_function}.

\subsubsection{Sharing State}

In React, there are various methods for sharing state, with one popular approach being to lift the state up \textit{(\autoref{fig:state_lifted})} \cite{react:sharing_state}. This method aligns with the principle of a \say{\textsl{single source of truth}} \cite{enwiki:single_source_of_truth}, meaning that while the state isn't confined to just one place, there's a central component responsible for managing it. This eliminates the need for duplicating state across components, thus reducing error-prone practices.

However, this approach has its drawbacks. For nested components, the state must be passed down through each level, leading to code bloat and unnecessary dependencies between components. To address this issue, a context provider can be employed. This provider enables all nested children, regardless of depth, to access and respond to the state without requiring explicit prop drilling or predefined component structures \textit{(\autoref{fig:context_provider})} \cite{react:passing_data_deeply_with_context}.

\begin{figure}[ht]
    \centering
    \begin{minipage}[t]{0.3\textwidth}
        \centering
        \includesvg[height=12em]{images/experiment/react/each_component_its_own_state}
        \caption{per component state}
        \label{fig:state_regular}
    \end{minipage}
    \hfill
    \begin{minipage}[t]{0.3\textwidth}
        \centering
        \includesvg[height=12em]{images/experiment/react/state_lifted}
        \caption{state lifted up \textit{(shared state)}}
        \label{fig:state_lifted}
    \end{minipage}
    \hfill
    \begin{minipage}[t]{0.3\textwidth}
        \centering
        \includesvg[height=12em]{images/experiment/react/context_provider}
        \caption{context provides state to all}
        \label{fig:context_provider}
    \end{minipage}
\end{figure}

\subsection{Redux}
\label{subsec:redux}

Incorporating context to manage state is useful, but it's typically confined to the parent component. In most projects, developers opt to place the global state in the root \textit{(app)} component \textit{(\autoref{fig:multiple_root_providers})}. While this approach suffices for a few states, it becomes unwieldy when multiple states need to be shared.

To mitigate this challenge, developers often turn to Redux, a state management library. Redux operates akin to a global context, with each state, termed a \say{store}, linked directly to the Redux provider \textit{(\autoref{fig:redux})} \cite{redux:overview}. This setup ensures that each state maintains its own logic and adheres to consistent, standardized definitions.

\begin{figure}[ht]
    \centering
    \begin{minipage}[t]{0.3\textwidth}
        \centering
        \includesvg[height=14em]{images/experiment/react/multiple_root_providers}
        \caption{several providers provide state to all}
        \label{fig:multiple_root_providers}
    \end{minipage}
    \hfill
    \begin{minipage}[t]{0.6\textwidth}
        \centering
        \includesvg[height=12em]{images/experiment/react/redux}
        \caption{several Redux stores provide state to all}
        \label{fig:redux}
    \end{minipage}
\end{figure}

\subsection{Strengths}
\label{subsec:react:strengths}
\begin{itemize}
    \item \acrshort{jsx} is intuitive and easy to write
    \item big ecosystem
    \item well-known and used among developers
    \item documentation is great \textit{(includes interactive examples)}
    \item big community \textit{(over 230,000 members on Discord)}
    \item lots of enterprise usage, even in the Fortune 500 companies \cite{buildin:react_companies}
    \item not that performant compared to native JS, but a one-click addon called \textsl{Million} \cite{million} makes this problem go away
    \item professional support is widely available through 3rd parties
\end{itemize}

\subsection{Weaknesses}
\label{subsec:react:weaknesses}
\begin{itemize}
    \item \acrshort{vdom} overhead
    \item Component tries can cause unexpected behavior
    \item client-side rendered \textit{(solution: Next.js or canary \say{use server} directive \cite{react:use_server_directive})}
    \item State management can easily become complex in bigger components.
    \item basic dependencies \textit{(React, Redux, \acrshort{i18n})} that get sent to the client are about ~12MB which is relatively big and can slow down the initial load
\end{itemize}

\subsection{Scores}
\label{subsec:react:scores}

\begin{table}[H]
    \centering
    \begin{tabular}{|l|l|}
        \hline
        \textbf{Method}            & \textbf{Score}                                    \\
        \hline
        Easiness/speed to learn    & 1                                                 \\ \hline
        State management           & 0.5                                               \\ \hline
        Boilerplate                & 0.5 \textit{(components: 1, state management: 0)} \\ \hline
        \acrshort{api} integration & 0 \textit{(default browser fetch \acrshort{api})} \\ \hline
    \end{tabular}
    \caption{complexity}
    \label{tab:react:complexity}
\end{table}

\itodo{relative performance}

\begin{table}[H]
    \centering
    \begin{tabular}{|l|l|}
        \hline
        \textbf{Method}                          & \textbf{Score}                                     \\
        \hline
        Community                                & 1 \textit{(\autoref{tab:metrics:community})}       \\ \hline
        Professional Support                     & 1                                                  \\ \hline
        Documentation (interactive walkthrough?) & 0.5                                                \\ \hline
        Ecosystem                                & 0.9 \textit{(\autoref{tab:metrics:react:ratings})} \\ \hline
        Usage by other enterprises               & 1                                                  \\ \hline
        \textit{(added)} Size of solution        & 0.5                                                \\ \hline
        Relative performance of solution         & -                                                  \\ \hline
        Complexity                               & 0.5 \textit{(\autoref{tab:react:complexity})}      \\ \hline
        Server Side Rendered (SSR)               & 0.5                                                \\ \hline
    \end{tabular}
    \caption{React general scores}
    \label{tab:react:scores}
\end{table}