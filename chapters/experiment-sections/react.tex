\section{React}
\label{sec:react}

\subsection{Overview}
\label{subsec:react:overview}

React is a \gls{library} created by Meta \textit{(originally known as Facebook)}. Its recommended usage is in combination with \acrshort{jsx} \textit{(see \ref{subsec:jsx})}. 
The library is primarily intended for the render layer and does not include native support for features such as routing and \acrshort{i18n}. However, this does not mean that you cannot easily incorporate these features, as the library is part of a vast community ecosystem that includes many high-quality packages that specialize in various areas.

React uses a different approach than plain JavaScript by utilizing the \acrshort{vdom} \cite{react:vdom} \textit{(see \ref{subsec:vdom})} to manage content instead of directly manipulating the \acrshort{dom}. It attempts to detect changes using the reconciliation algorithm \textit{(see \ref{subsec:react:reconciliation_algorithm})} in the browser at runtime \cite{react:reconciliation}.

\subsection{JSX}
\label{subsec:jsx}

\acrfull{jsx} is ...

\subsection{Virtual DOM}
\label{subsec:vdom}

The \acrshort{vdom} is...

\itodo{write about vdom}


\subsection{Reconciliation Algorithm}
\label{subsec:react:reconciliation_algorithm}
\itodo{write about the reconciliation algorithm}

\subsection{Redux}
\label{subsec:redux}
\itodo{write about redux}

\subsection{Strengths}
\label{subsec:react:strengths}
\begin{itemize}
    \item strength! \todo{write strengths}
\end{itemize}

\subsection{Weaknesses}
\label{subsec:react:weaknesses}
\begin{itemize}
    \item weakness! :( \todo{write weaknesses}
\end{itemize}