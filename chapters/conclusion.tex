\chapter{Conclusion}

This study examined various methods for building web applications, evaluating their strengths and weaknesses to identify the most suitable solution for different types of projects. The usability of each framework was also tested to provide a comprehensive evaluation.

\section{Summary of Findings}

The research indicates that Svelte, Vue, and React are top choices for general use, with each framework offering unique advantages.

\begin{enumerate}
    \item Svelte scored the highest overall, being straightforward to use for simple applications but requiring improvements in state management for more complex projects.
    \item Vue is particularly suitable for non-\acrshort{ssr} applications and offers a seamless developer experience.
    \item React has the largest ecosystem and community support, making it versatile for various projects.
    \item Angular offers a unique developer experience but generally lags behind its competitors.
    \item Lit excels in cross-framework components or enhancing plain HTML sites.
    \item Hilla introduces a novel frontend-backend communication concept and could pair well with React if current issues are resolved.
\end{enumerate}

\section{Implications}

The choice of a web framework should be tailored to the specific needs of the project. For non-\acrshort{ssr}, while React's extensive ecosystem makes it versatile. Angular offers unique features for the developer experience but needs to catch up in other areas.

\section{Limitations}

This study focused on popular frameworks and general use cases. The rapidly evolving web development landscape means that new features could change the current standings soon. Additionally, \acrshort{dx} can be subjective and may vary among individuals and teams.

\section{Future Research}

Future research should explore the long-term performance and scalability of these frameworks in different environments. Investigating the impact of new features and updates on usability and efficiency would be beneficial. A deeper analysis of specific use cases, such as e-commerce or content management systems, could provide more targeted recommendations.

\section{Concluding Remarks}

Selecting a web development framework should consider the specific needs and goals of the project. By understanding the specific advantages and limitations of each framework, developers can make informed decisions that enhance their development experience. This study highlights the importance of continuous evaluation and adaptation in the fast-paced field of web development.
