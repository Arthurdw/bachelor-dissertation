\chapter{Introduction}

\section{General}
In web development, there are numerous solutions for creating websites. These solutions streamline website creation by simplifying complex tasks, standardizing and abstracting away common tasks, increasing \acrshort{dx} and ultimately \acrshort{ux}.

\subsubsection{The Paradox of Choice}
The abundance of web solutions/frameworks presents a challenge known as the paradox of choice \cite{the_decision_lab:paradox_of_choice}. This challenge is particularly pertinent for enterprises, which face additional criteria such as release cycles, licensing, support, state management complexities, backing, and longevity — all crucial factors considering the long-term maintenance requirements of the software.

\subsubsection{Common Challenge}

Whether you're a \textit{(frontend)} Software Engineer, Project Manager, Technical Architect, Startup Founder, or simply someone intrigued by the web, you've likely encountered this dilemma.

\subsubsection{Collaborative Project}

This dissertation is a collaboration with the IT department of H. Essers, a major European Transport and Logistics company headquartered in Genk, Belgium. Their objective is to develop custom software solutions to optimize business processes. The department mainly comprises Java and IBM AS/400 \textit{(now called \say{IBM i} \cite{enwiki:ibm_i})} developers. They utilize the Vaadin \gls{full_stack} \gls{framework} for Java. Despite its advantages, the team has faced limitations within Vaadin that require more effort than initially anticipated.

\section{The problem}

User interfaces are crucial components of any application, and although websites have been around for many years, the industry is constantly evolving. Nowadays, development teams have a variety of web frameworks, architectures, and principles to choose from, making it difficult to decide which one is the best for a given task. This research aims to provide an answer to this difficult question.

\section{Research question}

What is the most suitable web solution for what kind of application?

\subsection{Web Solutions}
\label{subsec:intro:research:web_solutions}

Research has been conducted into:

\begin{table}[H]
    \centering
    \begin{tabular}{|l|l|l|}
        \hline
        Solution & Year Released & Version Reviewed \\
        \hline
        React    & 2022          & \texttt{18.2.0}  \\ \hline
        Vue      & 2024          & \texttt{3.4.15}  \\ \hline
        Svelte   & 2024          & \texttt{4.2.12}  \\ \hline
        Angular  & 2024          & \texttt{17.3.0}  \\ \hline
        Lit      & 2024          & \texttt{3.1.2}   \\ \hline
        Hilla    & 2024          & \texttt{2.5.7}   \\ \hline
        % HTMX        & unk          & \texttt{unk}   \\ \hline
        % Thymeleaf   & unk          & \texttt{unk}   \\ \hline
        % Alpinejs    & unk          & \texttt{unk}   \\ \hline
        % Go Templ    & unk          & \texttt{unk}   \\ \hline
        % Leptos      & unk          & \texttt{unk}   \\ \hline
    \end{tabular}
    \caption{Researched Solutions}
    \label{tab:researched_solutions}
\end{table}

\section{Experiment}

The same project was built in each solution to ensure equal and objective evaluation.

\subsection{Project Requirements}
\label{subsec:intro:experiment:project_requirements}

These requirements will provide valuable insights. The assessment is conducted objectively using the details stated in \ref{subsec:intro:experiment:evaluation}.

\begin{itemize}
    \item General Layout \textit{(see \autoref{fig:requirements:layout})}
    \item Interactive Search \textit{(with URL query reflection)} \textit{(see \autoref{fig:requirements:search}})
    \item \textit{(data)} Grid \textit{(see \autoref{fig:requirements:grid}})
    \item \textit{(data)} Grid in \textit{(data)} Grid \textit{(see \autoref{fig:requirements:grid_in_grid}})
    \item Normal Forms \textit{(with validation)} \textit{(see \autoref{fig:requirements:form_1} and \autoref{fig:requirements:form_2}})
    \item Wizard Forms \textit{(see \autoref{fig:requirements:wizard_1} and \autoref{fig:requirements:wizard_2}})
    \item Internationalization
    \item Drag and Drop \textit{(see \autoref{fig:requirements:dnd}})
    \item Progressive Loading
    \item Global State Management and Reactions
    \item Reflective Routing \textit{(see \autoref{fig:requirements:routing_1}, \autoref{fig:requirements:routing_2}, \autoref{fig:requirements:routing_3}, \autoref{fig:requirements:routing_4}, and \autoref{fig:requirements:routing_5})}
\end{itemize}

\subsection{Evaluation}
\label{subsec:intro:experiment:evaluation}

Because the research question is broad, it will be answered by dividing it into smaller evaluation points, which are ranked objectively using a suitable method.

\begin{itemize}
    \item Community \footnote{Points by size, see \ref{tab:metrics:community}}
    \item Professional Support \footnote{\Gls{predicate} \label{fn:predicate}}
    \item Documentation \textit{(interactive?)} \footnote{Likert Scale \label{fn:likert_scale}}
    \item Ecosystem \footnote{Points are 70\% by quality and 30\% by size, see \ref{app:metrics:ecosystem:quality} and \ref{app:metrics:ecosystem:size} respectively}
    \item Usage by other enterprises \footref{fn:likert_scale}
          % \item \textit{(added)} Size of solution \footnote{Likert Scale, but smaller is better}
          % \item Relative performance of solution \footref{fn:likert_scale}
    \item Complexity \footnote{Evaluated using the Likert Scale by easiness/speed to learn, state management, boilerplate, and API integration}
    \item \acrfull{ssr} \footnote{Likert Scale, \textsc{medium/ok} being available through a well-supported and known extension}$^,$ \footnote{\acrshort{ssr} is better for \acrshort{seo} dependent applications}
\end{itemize}

\subsubsection{Likert Scale}

Some evaluation will be done using a Likert Scale \cite{enwiki:likert_scale}, with the values:

\begin{center}
    \textsc{bad/not present} < \textsc{medium/ok} < \textsc{great}.
\end{center}
