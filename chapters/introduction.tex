\chapter{Introduction}
\section{General}
In the world of the web, there are a lot of web frameworks. A web framework allows you to write websites in a more developer-friendly (DX) way, by abstracting lower-level or repetitive concepts away, creating several utilities for improving developer productivity and satisfaction. 

\medskip

There are a lot of web frameworks, all promising to be great but this can cause the paradox of choice \cite{the_decision_lab:paradox_of_choice}, which is especially true for enterprises as they impose extra requirements, like release cycles, licensing, support, complex state management, backers, and longevity; as the software will still have to be maintained 10 years from now.

\medskip

If you are a \textit{(frontend)} Software Engineer, Project Manager, Technical Architect, Startup Founder, or just generally interested in the web you have most likely faced the same hurdle.
This thesis was created in collaboration with the IT department of H. Essers, one of the biggest Transport and Logistics companies in Europe, based in Genk \textit{(Belgium)}. Their task is to create custom software to optimize business processes. This department consists primarily of Java and IBM AS/400 \textit{(now known as "IBM i" \cite{enwiki:ibm_i})} developers. On the Java side, they are currently using the \gls{full_stack} framework Vaadin; this is all in the Java programming language. Despite Vaadin's benefits, the team has encountered limitations within the framework that demand more effort than initially anticipated for certain actions.


\section{The problem}
\itodo{Write this}
Specific limiting scope. The specific problem statement you are addressing. 

\section{Research question}
\itodo{Write this}
Based on the above problem statement, literally formulate the concrete research question. Clarify that you will seek an answer to this throughout the bachelor dissertation.

\section{Experiment}
\itodo{Write this}
What methods will you use to answer your research question? Briefly discuss your research design (type of research, data collection, data description and analysis method) here. This section is discussed in detail in the next chapter.
