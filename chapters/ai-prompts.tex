\def \thischapter{AI Engineering Prompts}

\chapter*{\thischapter}
\addcontentsline{toc}{chapter}{\thischapter}

\textit{Note: While LLMs are excellent, they are not flawless, so the prompts have been used as a guide rather than a definitive one. There has been no copy-pasting, but rather a rewriting process to match the AI's output.}


\section{Rewriting of text}
\footnotesize

Role: English Language Expert \\
Language: English \\
Context: Computer Science Thesis \\
Task: Rewrite the following text slightly according to the notes \\
Notes: \\
- Clarity: Ensure that the main ideas are clearly expressed and easy to understand. \\
- Simplicity: Use plain language and avoid jargon or complex terminology whenever possible. \\
- Conciseness: Trim unnecessary words or phrases to make the text more concise and to the point. \\
- Structure: Organize the text logically with clear headings, subheadings, and transitions between paragraphs. \\
- Consistency: Maintain consistent formatting, tone, and style throughout the text. \\
- Active Voice: Use active voice to make sentences more direct and engaging. \\
- Variety: Vary sentence length and structure to keep the reader's attention and avoid monotony. \\
- Clarity of Purpose: Ensure that each section or paragraph serves a clear purpose and contributes to \\
the overall message. \\
- Audience Awareness: Consider the needs and knowledge level of the target audience when \\
choosing language and examples. \\
- Visual Elements: Incorporate bullet points, lists, and visuals where appropriate to break up dense \\
text and improve readability. \\
- Transition Words: Use transition words and phrases to guide the reader through the text and connect ideas smoothly. \\
- Contextualization: Provide context or background information where necessary to help readers understand unfamiliar concepts or terms. \\
- Parallelism: Use parallel structure for lists or series of items to improve clarity and readability. \\
- Avoidance of Ambiguity: Clarify ambiguous terms or phrases to prevent confusion or misinterpretation. \\
- Proofreading: Correct any grammatical errors, typos, or inconsistencies. \\
- Output: The text is formatted using LaTeX, if there are any terms or acronyms that should be defined, define them and put them at the top separated from the text. \\
- Do not include LaTeX document boilerplate \\
Text: \\
<text here>

Role: English Language Expert
Language: English
Context: Computer Science Thesis
Task: Rewrite the following text slightly according to the notes
Notes:
- Clarity: Ensure that the main ideas are clearly expressed and easy to understand.
- Simplicity: Use plain language and avoid jargon or complex terminology whenever possible.
- Conciseness: Trim unnecessary words or phrases to make the text more concise and to the point.
- Structure: Organize the text logically with clear headings, subheadings, and transitions between paragraphs.
- Consistency: Maintain consistent formatting, tone, and style throughout the text.
- Active Voice: Use active voice to make sentences more direct and engaging.
- Variety: Vary sentence length and structure to keep the reader's attention and avoid monotony.
- Clarity of Purpose: Ensure that each section or paragraph serves a clear purpose and contributes to
the overall message.
- Audience Awareness: Consider the needs and knowledge level of the target audience when
choosing language and examples.
- Visual Elements: Incorporate bullet points, lists, and visuals where appropriate to break up dense
text and improve readability.
- Transition Words: Use transition words and phrases to guide the reader through the text and connect ideas smoothly.
- Contextualization: Provide context or background information where necessary to help readers understand unfamiliar concepts or terms.
- Parallelism: Use parallel structure for lists or series of items to improve clarity and readability.
- Avoidance of Ambiguity: Clarify ambiguous terms or phrases to prevent confusion or misinterpretation.
- Proofreading: Correct any grammatical errors, typos, or inconsistencies.
- Output: The text is formatted using LaTeX, if there are any terms or acronyms that should be defined, define them and put them at the top separated from the text.
- Do not include LaTeX document boilerplate
Text:
<text here>


Role: English Language Expert
Language: English
Context: Computer Science Thesis
Task: Rewrite the following text slightly according to the notes
Notes:
- Clarity: Ensure that the main ideas are clearly expressed and easy to understand.
- Simplicity: Use plain language and avoid jargon or complex terminology whenever possible.
- Conciseness: Trim unnecessary words or phrases to make the text more concise and to the point.
- Structure: Organize the text logically with clear headings, subheadings, and transitions between paragraphs.
- Consistency: Maintain consistent formatting, tone, and style throughout the text.
- Active Voice: Use active voice to make sentences more direct and engaging.
- Variety: Vary sentence length and structure to keep the reader's attention and avoid monotony.
- Clarity of Purpose: Ensure that each section or paragraph serves a clear purpose and contributes to
the overall message.
- Audience Awareness: Consider the needs and knowledge level of the target audience when
choosing language and examples.
- Visual Elements: Incorporate bullet points, lists, and visuals where appropriate to break up dense
text and improve readability.
- Transition Words: Use transition words and phrases to guide the reader through the text and connect ideas smoothly.
- Contextualization: Provide context or background information where necessary to help readers understand unfamiliar concepts or terms.
- Parallelism: Use parallel structure for lists or series of items to improve clarity and readability.
- Avoidance of Ambiguity: Clarify ambiguous terms or phrases to prevent confusion or misinterpretation.
- Proofreading: Correct any grammatical errors, typos, or inconsistencies.
- Output: The text is formatted using LaTeX, if there are any terms or acronyms that should be defined, define them and put them at the top separated from the text.
Text:
<text here>