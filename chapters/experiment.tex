\chapter{Experiment}

% structure to follow:
% framework
% -> overview
% -> strengths
% -> weaknesses

%%%%%%%%%%%%%%%%%%%%%%%%%%%%%%%%%%%%%%%%%%%%%%%%%%%%%%%%%%%%%%%%%%%%%%%
\section{React}
\label{sec:experiment:react}

\subsection{Overview}
\label{subsec:experiment:react:overview}

React is a \gls{library} created by Meta \textit{(originally known as Facebook)}. Its recommended usage is in combination with \acrshort{jsx}. 

\vspace{1cm}

React is a library, not a framework. This means that it does not contain all required tools to create a complete application. It mainly focuses itself on the render layer; and therefore doesn't have built in support for things like routing and \acrshort{i18n}. Altough

\subsection{JSX}
\label{subsec:jsx}

\acrfull{jsx} is ...

\subsection{Virtual DOM}
\label{subsec:vdom}

The \acrshort{vdom} is...

\itodo{write about vdom}

\subsection{Reconciliation Algorithm}
\label{subsec:experiment:react:reconciliation_algorithm}

\itodo{write about the reconciliation algorithm}

\subsection{Strengths}
\label{subsec:experiment:react:strengths}

\begin{itemize}
    \item strength!
\end{itemize}

\subsection{Weaknesses}
\label{subsec:experiment:react:weaknesses}

\begin{itemize}
    \item weakness! :(
\end{itemize}
%%%%%%%%%%%%%%%%%%%%%%%%%%%%%%%%%%%%%%%%%%%%%%%%%%%%%%%%%%%%%%%%%%%%%%%

%%%%%%%%%%%%%%%%%%%%%%%%%%%%%%%%%%%%%%%%%%%%%%%%%%%%%%%%%%%%%%%%%%%%%%%
\section{Vue}
\label{sec:experiment:vue}

\subsection{Overview}
\label{subsec:experiment:vue:overview}

\itodo{write this}

\subsection{Strengths}
\label{subsec:experiment:vue:strengths}

\begin{itemize}
    \item strength!
\end{itemize}

\subsection{Weaknesses}
\label{subsec:experiment:vue:weaknesses}

\begin{itemize}
    \item weakness! :(
\end{itemize}
%%%%%%%%%%%%%%%%%%%%%%%%%%%%%%%%%%%%%%%%%%%%%%%%%%%%%%%%%%%%%%%%%%%%%%%

\vspace{5cm}

This uses the \acrshort{vdom}!


\icomment{Estimated outline}
\begin{enumerate}
    \item What will we build and how?
    \item React short overview (what, by who, how, strengths \& weaknesses)
    \item Explaining the VDOM
    \item React Reconciliation Algorithm
    \item Vue short overview (what, by who, how, strengths \& weaknesses)
    \item Svelte short overview (what, by who, how, strengths \& weaknesses)
    \item Angular short overview (what, by who, how, strengths \& weaknesses)
    \item Hilla short overview (what, by who, how, strengths \& weaknesses)
    \item Do we even need a framework?
    \item Don’t write JS/TS solutions (HTMX, Hyperscript, …) \todo{better title for this}
    \item For enterprises (release cycles, stability, support, licenses, scalability, …) 
\end{enumerate}

\itodo{Write the experiment}
In this part of your bachelor's dissertation, you describe your experiment. Or put another way, how did you gain knowledge to give your TEDTalk/final presentation. 
This is the body of your report, where you make the difference compared to existing literature. Also make sure it is clear what you have made/added compared to existing material.
Make sure the following issues are covered throughout the different chapters:
\begin{itemize}
    \item the methodology
    \item justified choices of technology/software/procedure etc., 
    \item results, 
    \item critical analysis of the results. 
\end{itemize}
The above items are not literal chapters but should be interwoven throughout the body of your documentation report.
Make sufficient use of figures and visualisations (do not forget citations). Make sure you have a nice coherent whole, with clear structure and a smooth readability. Don't be afraid to use AI tools for this. Do not forget to include your AI prompts at the back of this report.
Too many details that would distract from the story of your bachelor's dissertation should be relegated to an appendix. (E.g. installation procedure, pieces of code.) 
