\newglossaryentry{full_stack}{
    name=full-stack,
    description={Encompasses the complete spectrum of web development, including both frontend and backend components, typically unified within a single codebase and language}
}

\newglossaryentry{developer_experience}{
    name=developer experience,
    description={refers to the overall quality of interaction and satisfaction developers encounter while using tools, libraries, frameworks, or platforms to build software solutions. It encompasses various aspects such as ease of use, clarity of documentation, efficiency of workflows, availability of support, and the overall enjoyment of the development process. A positive \acrshort{dx} contributes to increased productivity, reduced frustration, and greater motivation among developers, ultimately creating higher-quality software products}
}

\newacronym{dx}{DX}{\gls{developer_experience}}

\newglossaryentry{user_experience}{
    name=user experience,
    description={user experience refers to how people feel and interact with a product, service, or system. It encompasses aspects such as ease of use, visual appeal, and the emotions it evokes. A positive \acrshort{ux} keeps people satisfied, interested, and returning for more}
}

\newacronym{ux}{UX}{\gls{user_experience}}

\newglossaryentry{boolean}{
    name=boolean,
    description=a value which is either true or false \textit{(0 or 1)}
}

\newglossaryentry{predicate}{
    name=predicate,
    description=something that results in a \gls{boolean}
}

\newglossaryentry{document_object_model}{
    name=document object model,
    description={\say{connects web pages to scripts or programming languages by representing the structure of a document—such as the HTML representing a web page—in memory} \cite{mdn:dom}}
}

\newacronym{dom}{DOM}{\gls{document_object_model}}

\newacronym{vdom}{VDOM}{Virtual \acrshort{dom}}

\newglossaryentry{internationalisation}{
    name=internationalisation,
    description={is the process of designing applications to support multiple languages and cultures without altering the core code. It involves enabling language support, considering regional formats, supporting diverse character sets, managing content for localization, and adapting the user interface. Techniques include using resource bundles, language negotiation, dynamic content rendering, and ensuring Unicode compliance. Internationalization expands reach, improves user experience, and ensures regulatory compliance, making applications accessible to diverse global audiences}
}

\newacronym{i18n}{I18N}{\gls{internationalisation}}

\newglossaryentry{library}{
    name=library,
    description={a collection of pre-existing code that can be utilized to create new code \cite{codeacademy:what_is_a_framework}}
}

\newglossaryentry{framework}{
    name=framework,
    description={a supporting structure that requires specificity. You must follow the pattern of the framework. It is essentially a skeleton that is in control of your code. With a \gls{library}, your code is in control \cite{codeacademy:what_is_a_framework}}
}

\newacronym{xml}{XML}{Extensible Markup Language}

\newacronym{html}{HTML}{Hypertext Markup Language}

\newacronym{jsx}{JSX}{JavaScript \acrshort{xml}}

\newacronym{ui}{ui}{user interface}

\newacronym{js}{JS}{JavaScript}

\newacronym{json}{JSON}{JavaScript Object Notation}

\newacronym{seo}{SEO}{Search Engine Optimalisation}

\newacronym{ssr}{SSR}{Server Side Rendered}

\newacronym{api}{API}{Application Programming Interface}